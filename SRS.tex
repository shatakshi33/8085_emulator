\documentclass[12pt]{article}
\begin{document}
	\title{SOFTWARE REQUIREMENT SPECIFICATIONS (SRS)}
	\maketitle
	\section{About Software}
	This is a very basic implementation of 8085 microprocessor with a basic debugger madein C++.\\
	\\ 
	It includes some basic instructions like\\ 
	Load and Store - MOV, MVI, LXI, LDA, STA, LHLD, SHLD, XCHG\\
	Arithmetic - ADD, SUB, INR, DCR, INX, DCX, DAD, SUI\\
	Logical - CMA, CMP\\
	Branching - JMP, JC, JZ, JNZ\\
	One additional command SET should be madeto set data into\\ valid memory location (Eg SET 2500 0A).
	\section{Debugger}
	We have implemented a basic debugger with following commands:-
	1. break or b'line no':-It will set break point on given line number.\\
	2. run or r:-Run the program one until it ends or breakpoint is encountered.\\
	3. step or s:- It will run the program one instruction at a time.\\
	4. print or p:- It prints the value of register or memory location. For ex p A print the value of  x2500 will print the value of register A. p x2500 will print the value at memory location x2500 if any.\\
	5. quit or :- quits the debugger.\\
	help:- will show all the commands of debugger. \\
	\\
	6. The software output the value of the followings, if 8085 program is correct:-\\
	7. The program will display contents of Registers A,B,C,D,E,H,L ,flag Registers and used  memory locations only after the\\ execution of the program.\\
	8085 have 7 registers A,B,C,D,E,H,L and 5 flag registers which are following:-\\
	Sign Flag (Sf), Zero Flag(ZF), Auxiliary Flag(AF), Parity Flag(PF), Carry Flag(CF)\\
	9. The Software also include command line arguments so you can run it through cmd prompt also.\\
	
	\section{Instructions} 
	1.The program must be end with HLT command.\\
	2.Input should be in Capital letters only.\\
	3.Some inputs files are made avalaible in Debug folder in bin.\\
	4.This is an application program so run through .exe file in same debug folder as above.\\
	\section{Learning Outcomes}
	We get to learn about 8085 microprocessor, Object oriented programming with C++, an idea about how gdb debugger works and idea about project development.\\
	\section{Contacts}
	 This application is a joint contribution of\\
	 Ritesh Ranjan (B.tech(CS) 5th Sem Sec B )\\
	 email - riteshranjan5245147@gmail.com\\
	 Shatakshi Agarwal (B.tech(CS) 5th Sem Sec B )\\
	 shatakshimittal33@gmail.com\\
	 Forany query,Feel free to drop an email. \\
 \end{document}